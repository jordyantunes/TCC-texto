% RESUMO--------------------------------------------------------------------------------

\begin{resumo}[RESUMO]
\begin{SingleSpacing}

% Não altere esta seção do texto--------------------------------------------------------
\imprimirautorcitacao. \imprimirtitulo. \imprimirdata. \pageref {LastPage} f. \imprimirprojeto\ – \imprimirprograma, \imprimirinstituicao. \imprimirlocal, \imprimirdata.\\
%---------------------------------------------------------------------------------------

% Atualmente, a quantidade de vídeos presentes nas plataformas online só tende a crescer, provocando aflição em muitos criadores de conteúdo digital, que temem a possibilidade de cópia de seu material. 

Em plataformas digitais de compartilhamento de vídeo, é comum a violação de direitos autorais através do upload de material com copyright. Muitos fraudadores dificultam a identificação, aplicando modificações nos vídeos originais para que não sejam identificados tão facilmente --- modificações essas conhecidas como ataques. Devido ao número elevado de vídeos na internet, a identificação manual de fraudes é inviável, abrindo espaço para processos automatizados de reconhecimento de cópias. Para realizar estes processos, diversos algoritmos para cálculo de assinaturas digitais foram desenvolvidos. Neste trabalho, foram selecionados e comparadas sete assinaturas de vídeo, baseadas em informações temporais e espaciais, para análise da capacidade de reconhecimento de fraudes. Em nossos experimentos, verificamos que assinaturas com características temporais são menos eficientes, sendo sensíveis a alterações temporais e fotométricas, enquanto que algoritmos espaciais são sensíveis a alterações temporais. \\

\textbf{Palavras-chave}: Assinatura Digital de Vídeo, Descritor de vídeo, Cópias de Vídeos

\end{SingleSpacing}
\end{resumo}

% OBSERVAÇÕES---------------------------------------------------------------------------
% Altere o texto inserindo o Resumo do seu trabalho.
% Escolha de 3 a 5 palavras ou termos que descrevam bem o seu trabalho 

