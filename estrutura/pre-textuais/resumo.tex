% RESUMO--------------------------------------------------------------------------------

\begin{resumo}[RESUMO]
\begin{SingleSpacing}

% Não altere esta seção do texto--------------------------------------------------------
\imprimirautorcitacao. \imprimirtitulo. \imprimirdata. \pageref {LastPage} f. \imprimirprojeto\ – \imprimirprograma, \imprimirinstituicao. \imprimirlocal, \imprimirdata.\\
%---------------------------------------------------------------------------------------

Atualmente, a quantidade de vídeos presentes nas plataformas online só tende a crescer, provocando aflição em muitos criadores de conteúdo digital, que temem a possibilidade de cópia de seu material.  Muitos fraudadores ainda dificultam a identificação, aplicando modificações nos vídeos originais para que não sejam identificados tão facilmente. O número elevado de vídeos na internet torna a identificação de fraudes manual ineficaz, abrindo espaço para processos automatizados de reconhecimento de cópias. Para realizar estes processos, diversos algoritmos de descrição de imagem são desenvolvidos, visando detectar cópias nas quais ataques de imagem foram aplicados. Foram selecionados e comparados sete algoritmos de descrição de assinatura, para análise da capacidade de reconhecimento de fraudes, mostrando que algoritmos com características temporais são menos eficientes, sendo sensíveis a alterações temporais e fotométricas, enquanto que outros algoritmos são mais sensíveis apenas a alterações temporais. \\

\textbf{Palavras-chave}: Assinatura Digital de Vídeo, Descritor, Cópias de Vídeos

\end{SingleSpacing}
\end{resumo}

% OBSERVAÇÕES---------------------------------------------------------------------------
% Altere o texto inserindo o Resumo do seu trabalho.
% Escolha de 3 a 5 palavras ou termos que descrevam bem o seu trabalho 

