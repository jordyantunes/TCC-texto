% RESUMO--------------------------------------------------------------------------------

\begin{resumo}[RESUMO]
\begin{SingleSpacing}

% Não altere esta seção do texto--------------------------------------------------------
\imprimirautorcitacao. \imprimirtitulo. \imprimirdata. \pageref {LastPage} f. \imprimirprojeto\ – \imprimirprograma, \imprimirinstituicao. \imprimirlocal, \imprimirdata.\\
%---------------------------------------------------------------------------------------

Tradicionalmente a esteganálise é feita utilizando descritores criados através do conhecimento específico da área. Estudos recentes mostraram a eficácia do uso da aprendizagem de características em esteganálise. Esse trabalho propõe uma metodologia de comparação dessas duas abordagens em diferentes ambientes estegoanalíticos. Com essa metodologia foi possível chegar em discernimentos acerca da natureza das abordagens em relação a ambientações de esteganálise. A abordagem tradicional obteve resultados melhores em ambientes de esteganálise direcionada, por outro lado, a abordagem com aprendizagem profunda (\textit{deep learning}) aparenta ter um poder maior de generalização, assim, tendo resultados melhores em ambientes de esteganálise cega.\\

\textbf{Palavras-chave}: Esteganografia. Esteganálise. Aprendizagem Profunda.

\end{SingleSpacing}
\end{resumo}

% OBSERVAÇÕES---------------------------------------------------------------------------
% Altere o texto inserindo o Resumo do seu trabalho.
% Escolha de 3 a 5 palavras ou termos que descrevam bem o seu trabalho 

