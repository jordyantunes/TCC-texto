\chapter{Introdução}
\label{chap:introducao}

A cada minuto é realizado o upload de 300 horas de vídeo apenas no \textit{youtube.com}, segundo os dados da pesquisa realizada pelo site statisticbrain.com em 2016. O \textit{YouTube} é um dos serviços de hospedagem de vídeos mais conhecidos na internet, mas é importante notar que existem diversas outras empresas que oferecem serviços similares, como por exemplo Vimeo, Twitch, DailyMotion, etc.

% 2 Parágrafos com o problema (justificativa)
Suponha que essas plataformas de compartilhamento de vídeos recebam pedidos vindos de criadores de conteúdo digital (como estúdios de cinema, cineastas independentes ou ``vloggers'') solicitando a retirada de determinados vídeos da plataforma alegando o infringimento de direitos autorais. Segundo estes criadores, seus filmes estão sendo reproduzidos parcial ou integralmente na plataforma sem autorização. Enquanto o número de pedidos de retirada é baixo, não há grandes problemas em definir se o caso realmente se trata de uma cópia, pois um humano pode analisar os dois vídeos e definir se realmente um é cópia do outro. Entretanto, à medida que o número de pedidos cresce e levando em conta a quantidade de vídeos enviados a cada minuto para as plataformas, torna-se  inviável a realização desse processo de forma manual. 

Para realizar essa tarefa de forma automática, podem ser utilizados métodos de Detecção de Cópias Baseada em Conteúdo para a criação uma base de dados de assinaturas, para então ser realizada a busca e verificação de cópias.

Para realizar essa tarefa de forma automática, propomos a utilização de programas que têm a capacidade 

% 1 Parágrafo de contextualização
A Detecção de Cópias Baseada em Conteúdo é um método utilizado para a proteção de propriedade intelectual de mídias digitais, que consiste em extrair uma assinatura da mídia original e de uma mídia de teste e compará-las para identificar se as duas mídias têm o mesmo conteúdo ou não. Essa abordagem assume que a mídia contenha alguma informação única que possa ser utilizada para identificar uma cópia \citeauthor{kim2005spatiotemporal}. 

% Devemos incluir outros problemas?
% - identificação de conteúdo sendo reproduzido na TV/contar quantidade de exibições de uma propaganda
% - identificar e parar a disseminação de conteúdo proibido/impróprio



% 1 parágrafo mencionando estado da arte
Um dos serviços comerciais mais utilizados para este fim é o \textit{Content ID} da Audible Magic. O método utilizado cria uma assinatura baseando-se no aúdio de uma mídia digital (podendo então ser utilizado tanto para vídeos, quanto para músicas), e disponibiliza um banco de dados global de assinaturas de conteúdos protegidos por propriedade intelectual. Este serviço é usado por grandes corporações como Facebook, SoundCloud e Vimeo \citeauthor{audiblemagic}.

No entanto, o uso do áudio para a criação de assinaturas pode não ser o melhor atributo utilizado na criação de uma assinatura, pois vídeos (principalmente longas-metragens) tendem a ter versões dubladas na língua de cada país onde são distribuídos. 

Para combater este problema, foram propostos inúmeros métodos que utilizam informações contidas nos quadros dos vídeos para a geração das assinaturas. Este trabalho propõe-se a fazer uma revisão da literatura de métodos para Detecção de Cópias Baseada em Conteúdo, selecionar seis candidatos que possuam abordagens distintas, compará-los e avaliar quais atributos do conteúdo digital devem ser escolhidos a fim de otimizar a acurácia na detecção de cópias. 

% Mencionar que é a continuação do TCC do Sílvio
% Explicar critérios de escolha - Local, global, espacial, temporal
% Falar da organização do trabalho 

\section{Objetivo Geral}
Este trabalho tem como objetivo comparar descritores para assinatura digital de vídeos.

\section{Objetivos Específicos}
\begin{itemize}
\item Implementar 6 algoritmos, locais e globais, para assinatura digital de vídeos.
\item Propor um novo descritor que combine características espaciais e temporais do vídeo.
\item Compilar uma base de vídeos para a realização dos testes e avaliação dos algoritmos.
\item Comparar os algoritmos com base na semelhança da assinatura e contrastar os métodos de comparação utilizados.
\end{itemize}