% 1 Parágrafo de contextualização
A Detecção de Cópias Baseada em Conteúdo é um método utilizado para a proteção de propriedade intelectual de mídias digitais. Ele consiste em extrair uma assinatura da mídia original e de uma mídia de teste, e então compará-las para identificar que se trata do mesmo conteúdo ou não. Esta abordagem assume que a mídia contenha alguma informação única que possa ser utilizada para identificar uma cópia \citeauthor{kim2005spatiotemporal}. 

% 2 Parágrafos com o problema (justificativa)
Suponha que uma plataforma online de publicação de vídeos (como YouTube ou Vimeo) receba vários pedidos vindos de criadores de conteúdo digital (como estúdios de cinema, cineastas independentes ou ``vloggers'') solicitando a retirada de vídeos da plataforma alegando o infringimento de direitos autorais. Segundo estes criadores, seus filmes estão sendo reproduzidos parcial ou integralmente na plataforma. 

Enquanto o número de pedidos é baixo, não há grandes problemas em definir se realmente se trata de uma cópia: Basta assistir ao vídeo original e à suposta cópia, e então comparar os conteúdos. A medida que o número de pedidos cresce ou que os a extensão dos vídeos seja cada vez mais longa (filmes tendem a ter pelo menos duas horas de duração), torna-se claramente inviável a realização desse processo de forma manual.

% Devemos incluir outros problemas?
% - identificação de conteúdo sendo reproduzido na TV/contar quantidade de exibições de uma propaganda
% - identificar e parar a disseminação de conteúdo proibido/impróprio

Para realizar esta tarefa de forma automática, podem ser utilizados métodos de Detecção de Cópias Baseada em Conteúdo para a criação uma base de dados de assinaturas, para então ser realizada a busca e verificação de cópias.

% 1 parágrafo mencionando estado da arte
Um dos serviços comerciais mais utilizados para este fim é o \textit{Content ID} da Audible Magic. O método utilizado cria uma assinatura baseando-se no aúdio de uma mídia digital (podendo então ser utilizado tanto para vídeos, quanto para músicas), e disponibiliza um banco de dados global de assinaturas de conteúdos protegidos por propriedade intelectual. Este serviço é usado por grandes corporações como Facebook, SoundCloud e Vimeo \citeauthor{audiblemagic}.

No entanto, o uso do áudio para a criação de assinaturas pode não ser o melhor atributo utilizado na criação de uma assinatura, pois vídeos (principalmente longas-metragens) tendem a ter versões dubladas na língua de cada país onde são distribuídos. 

Para combater este problema, foram propostos inúmeros métodos que utilizam informações contidas nos quadros dos vídeos para a geração das assinaturas. Este trabalho propõe-se a fazer uma revisão da literatura de métodos para Detecção de Cópias Baseada em Conteúdo, selecionar seis candidatos que possuam abordagens distintas, compará-los e avaliar quais atributos do conteúdo digital devem ser escolhidos a fim de otimizar a acurácia na detecção de cópias. 

% Mencionar que é a continuação do TCC do Sílvio
% Explicar critérios de escolha - Local, global, espacial, temporal
% Falar da organização do trabalho 