\chapter{Introdução}
\label{chap:introducao}

A cada minuto é realizado o \textit{upload} de aproximadamente 300 horas de vídeo apenas na plataforma \textit{YouTube\footnote{www.youtube.com/}}, segundo os dados da pesquisa realizada pelo site \textit{Statistic Brain\footnote{www.statisticbrain.com/youtube-statistics/}}, em 2016. É provável que essa estatística seja ainda mais expressiva atualmente, devido à popularização cada vez maior de dispositivos móveis e à democratização do acesso à internet. O \textit{YouTube} é um dos serviços de hospedagem de vídeos mais utilizados na internet, em conjunto com diversas outras empresas que oferecem serviços similares, como, por exemplo, \textit{Vimeo, Twitch, DailyMotion} e, mais recentemente o \textit{Facebook}.

% 2 Parágrafos com o problema (justificativa)
É comum que essas plataformas de compartilhamento de vídeos recebam pedidos vindos de criadores de conteúdo digital (como estúdios de cinema, cineastas independentes ou ``vloggers'') solicitando a retirada de determinados vídeos alegando o infringimento de direitos autorais. Segundo estes criadores, seus vídeos estão sendo reproduzidos parcial ou integralmente sem autorização, caracterizando uma situação de pirataria e apropriação indevida de conteúdo. 

Enquanto o número de pedidos de retirada é baixo não há grandes problemas em definir se o caso realmente se trata de uma cópia, pois um humano pode analisar o vídeo e definir se realmente se trata de um caso de plágio. Entretanto, à medida que o número de pedidos cresce, levando em conta a quantidade de \textit{uploads}, torna-se inviável a realização desse processo de forma manual, gerando um problema para as plataformas, que precisam buscar métodos automáticos de detecção de duplicatas de vídeos.

Uma das técnicas já adotadas pelas empresas de compartilhamento de vídeos é a detecção de duplicatas via áudio. Um exemplo é o serviço chamado \textit{Content ID} , que de acordo com a empresa \citeonline{audiblemagic}, disponibiliza comercialmente uma base de dados global para verificação de conteúdos protegidos por direitos autorais \cite{audiblemagic}. Tomando por exemplo o \textit{YouTube}, um dos seus sistemas de prevenção de cópias analisa o áudio de cada vídeo enviado \cite{youtubeblog}. O áudio é então comparado com uma base de dados para verificar se aquele conteúdo está de acordo com as políticas anti-pirataria da plataforma e, caso seja encontrada alguma irregularidade, o vídeo é rejeitado e a pessoa que realizou o \textit{upload} é notificada que está infringindo os termos de uso. Essa é abordagem eficaz para casos específicos, como a reprodução ilegal de um videoclipe ou a utilização ilícita de trilhas sonoras protegidas por lei. Porém, não é totalmente satisfatória para reconhecer se o conteúdo do vídeo está sendo utilizado indevidamente, justamente por utilizar o áudio para detectar duplicatas, não o vídeo. Outra característica negativa dessa técnica é que longas-metragens normalmente têm versões dubladas para cada país onde são distribuídas, aumentando o problema para detectar cópias.

%Existem serviços comerciais destinados a encontrar, de maneira automática, produções que infringem os direitos autorais. Um dos mais utilizados é o \textit{Content ID} da Audible Magic. O método utilizado cria uma assinatura baseanda no aúdio de uma mídia digital (podendo então ser utilizado tanto para vídeos, quanto para músicas), e disponibiliza um banco de dados global de assinaturas de conteúdos protegidos por propriedade intelectual. Este serviço é usado por grandes corporações como Facebook, SoundCloud e Vimeo, de acordo com  \citeauthor{audiblemagic}.

O problema de detecção de duplicatas se torna mais complexo quando os piratas utilizam técnicas de modificação nos vídeos, também conhecidas como ataques, fazendo com que sistemas de identificação mais especializados sejam necessários. Essas modificações podem ser sutis, como a remoção de alguns quadros do vídeo ou a alteração do formato de compressão, ou mais agressivas, como a modificação das cores, espelhamento, rotação e inserção de bordas nos vídeos. O desafio, entretanto, é que devido à natureza de cada ataque, pode se fazer necessária a utilização de diferentes formas de análise, já que um sistema para identificar um ataque de rotação pode ter dificuldade em detectar um ataque de remoção de quadros, por exemplo. 

Como uma possível solução complementar para esse problema, propomos um estudo sete algoritmos para identificar um vídeo em relação ao seu conteúdo \cite{hua2004robust}, \cite{lee2008robust}, \cite{cook2011efficient}, \cite{mao2015sceneframe}, \cite{kim2014rotation}, \cite{Dutta2013}, \cite{minetto2007reliable}. Essa identificação é feita através da extração de uma assinatura do vídeo, que deve ser robusta aos ataques mais comuns que um vídeo pode sofrer. O processo de produção das assinaturas varia de acordo com o algoritmo utilizado e as assinaturas são basicamente características locais, globais, espaciais ou temporais dos vídeos. 




%Nas próximas seções estão definidos conceitos utilizados neste trabalho, como quadro (\textit{frame}), vídeo e tomada, além da definição de assinatura de vídeo e algumas características consideradas importantes para a geração destas. Na seção \ref{sec:estadodaarte} são abordados os avanços no campo de detecção de cópias baseada em conteúdo.


    
    
%Para atacar esse problema, propomos a criação de uma base de dados de assinaturas de vídeos, ou \textit{fingerprints}, através da utilização de algoritmos que têm a capacidade de detectar o conteúdo do vídeo e gerar essa assinatura para representá-lo. Com a base de assinaturas, então, será possível comparará-las para identificar se existem assinaturas que são semelhantes o suficiente para afirmar se são cópias de outros vídeos ou não, ou identificar se existe alguma assinatura representando algum vídeo marcado para remoção devido aos direitos autorais.

% 1 Parágrafo de contextualização
%A Detecção de Cópias Baseada em Conteúdo é um método utilizado para a proteção de propriedade intelectual de mídias digitais, que consiste em extrair uma assinatura da mídia original e de uma mídia de teste e compará-las para identificar se as duas mídias têm o mesmo conteúdo ou não. Essa abordagem assume que a mídia contenha alguma informação única que possa ser utilizada para identificar uma cópia \citeauthor{kim2005spatiotemporal}. 

% Devemos incluir outros problemas?
% - identificação de conteúdo sendo reproduzido na TV/contar quantidade de exibições de uma propaganda
% - identificar e parar a disseminação de conteúdo proibido/impróprio




% Para combater o problema, foram propostos métodos que utilizam informações contidas nos quadros dos vídeos para a geração das assinaturas. Este trabalho propõe-se a fazer uma revisão da literatura de métodos para Detecção de Cópias Baseada em Conteúdo, selecionar seis candidatos que possuam abordagens distintas, compará-los e avaliar quais atributos do conteúdo digital devem ser escolhidos a fim de otimizar a acurácia na detecção de cópias. 

% Explicar critérios de escolha - Local, global, espacial, temporal
% Falar da organização do trabalho 

\section{Objetivo Geral}
\label{sec:objetivos}
Este trabalho tem como objetivo comparar descritores para assinatura digital de vídeos, e investigar a complementaridade de descritores espaciais em conjunto com uma proposta de descritor temporal.

\section{Objetivos Específicos}

Implementar sete algoritmos para gerar assinaturas digitais de vídeos. Criar um repositório com 1265 vídeos que sofrerão 14 tipos de ataques diferentes para testar a robustez dos algoritmos. As assinaturas dos vídeos e suas respectivas duplicadas serão registradas em  uma base de dados, que possibilitará a análise e extração de informações sobre a eficiência de cada método. %Por fim, comparar os algoritmos com base na semelhança das assinaturas e contrastar os métodos de comparação utilizados.

Parte do estudo será criar um repositório com 1265 vídeos que sofrerão 14 tipos de ataques diferentes para testar a robustez dos algoritmos. As assinaturas dos vídeos e suas respectivas duplicadas serão registradas em  uma base de dados, que possibilitará a análise e extração de informações sobre a eficiência de cada método.

\section{Estrutura do trabalho}
Este trabalho está estruturado como segue. No Capítulo 2 são apresentados conceitos básicos, uma introdução aos métodos de detecção de conteúdo em vídeos, os seis algoritmos selecionados para a geração de assinaturas e trabalhos relacionados. O Capítulo 3 detalha a base de vídeos utilizada, a forma como o experimento foi realizado e as métricas de comparação. No Capítulo 4 são apresentados os resultados experimentais. Finalmente, são apresentadas as conclusões do estudo e sugestões de trabalhos futuros.















