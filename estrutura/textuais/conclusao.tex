% CONCLUSÃO--------------------------------------------------------------------

\chapter{Conclusão}
\label{chap:conclusao}

Este trabalho apresenta uma revisão da literatura moderna sobre detecção de cópias de vídeo baseada em conteúdo. Este tema é cada vez mais importante devido a quantidade de conteúdo audio-visual sendo enviado para a internet a cada minuto e a facilidade de propagação ilegal de conteúdo protegido por direitos autorais. Para isso, foram realizados experimentos com as principais abordagens utilizadas atualmente a fim de descobrir seus pontos fracos e verificar se a combinação de abordagens de tipos diferentes pode melhorar a detecção de cópia baseada em conteúdo.

Nos resultados, a utilização de assinaturas temporais obteve o pior resultado em todos os testes, sendo especialmente sensíveis a distorções temporais e fotométricas. Enquanto isso, a assinaturas que utilizam características globais de um vídeo como luminância obtiveram desempenho razoável, mas foram altamente sensíveis a distorções fotométricas.

De modo geral, a assinatura Wavelets obteve os melhores resultados na classificação, conseguindo classificar corretamente a maioria dos casos de distorção fotométrica e temporal, mas sendo mais afetada por distorções de tipo geométrico. Mesmo com esta sensibilidade a ataques geométricos, no caso em que obteve os piores resultados (distorção crop), seu desempenho foi acima da média geral e ficou em segundo lugar na medida fmeasure (que combina as métricas de robustez e unicidade).

No teste de combinação de assinaturas, notou-se uma clara diminuição na robustez e unicidade de modo geral em comparação a aplicação de assinaturas de modo isolado. Ao invés de uma combinação das características que tornam cada tipo de assinatura robusto, a combinação somou a sensibilidade da assinatura temporal Camera Motion a todas as outras, piorando as métricas de robustez e unicidade em praticamente todos os casos.

Isso não quer dizer, no entanto, que a combinação de tipos assinatura não pode ser beneficial à detecção de cópias. Este trabalho contribuiu para a literatura de forma a encontrar empiricamente características de cada tipo assinatura que as torna robustas, abrindo caminho a criaçao de novas assinaturas que utilizam as propriedades de vídeos com maior poder discriminante.

\section{Trabalhos Futuros}
\label{sec:trabalhosFuturos}

// todo