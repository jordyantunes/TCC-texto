\section{Assinatura baseada em quadros de cena}

	Outra abordagem, proposta por \citeauthor{mao2015sceneframe}, é baseada na assinatura de quadros de cena. De acordo com os autores, os quadros de cena podem ser \textit{intraframes}, ou seja, quadros que iniciam tomadas, quanto \textit{interframes}, contanto que sigam as características descritas em \ref{sec:quadrocena}.
    
    O algoritmo fundamenta-se na ideia de que as chances de existirem cinco quadros de cena seguidos é extremamente baixa, por isso são selecionados apenas os cinco primeiros quadros de cena de um vídeo. A forma como estes são determinados é descrita no Algoritmo \ref{alg:sceneframe}.

\subsection{A extração de assinatura}

\input{dados/algoritmos/scene_frame.tex}

Primeiramente, todo quadro é pré-processado, seguindo uma série de passos que serão descritos a seguir:

\begin{enumerate}
	\item O componente de luminância é obtido
   	\item O quadro é recortado, utilizando-se apenas sua área central
    \item O quadro é redimensionado para o tamanho de $3/4$QCIF, ou seja, $(108\times132)$
\end{enumerate}

Após o processamento inicial, o quadro é então dividido em $144$ pedaços menores, de tamanho $(9\times11)$, cuja média de intensidade irá compor parte da assinatura deste quadro. Além dos $144$ valores, o descritor é composto também por $576$ elementos diferenciais, totalizando $720$ valores. Para obter esses elementos, cada fragmento é dividido em oito elementos menores, como mostra a Figura \ref{fig:divsceneframe}, e então é realizada a subtração de $a - b$, $c - d$, $e - f$ e $g - h$.

\begin{figure}[h]
	\centering
    \label{fig:divsceneframe}
	\includegraphics[width=\textwidth]{dados/figuras/divisaosceneframe.jpg}
    \caption{Divisão da imagem para cálculo dos elementos diferenciais. Referência: \citeauthor{mao2015sceneframe}}
\end{figure}

\subsection{Diminuição do espaço de memória utilizado}

O artigo também propõe uma alternativa para diminuir o espaço de memória utilizado para armazenar as assinaturas, visto que o banco de dados dos vídeos pode ser grande. Para isso, é proposta uma técnica chamada qualificação quaternária, na qual os valores são classificados de acordo com um \textit{threshold}.



