% INTRODUÇÃO-------------------------------------------------------------------

\chapter{Introdução}
\label{chap:introducao}

% Sugestão da professora Leyza
% 1 Parágrafo de contextualização
% 2 Parágrafos com o problema (justificativa)
% 1 parágrafo mencionando estado da arte
% 2/3 parágrafos com a solução (método proposto e resultados parciais)
% 1 parágrafo com organização do trabalho

Uma grande quantidade de conteúdo digital em vídeo está disponível em plataformas como YouTube e Vimeo. Em razão disso, há uma crescente demanda por meios de proteger a propriedade intelectual dessas produções \citeauthor{hua2004robust}. De acordo com \citeauthor{kim2005spatiotemporal} uma maneira de identificar cópias de vídeo é gerar uma assinatura digital para cada produção, através de um algoritmo, que poderá ser comparada com a base de dados de assinaturas de outros vídeos.

	Conforme a definição de \citeauthor{kim2005spatiotemporal}, a detecção de cópias baseada no conteúdo é uma abordagem que utiliza as informações de elementos presentes nas próprias mídias, como bordas, objetos e cores, para gerar uma assinatura discriminante e robusta. Essa técnica também é útil para monitorar e rastrear quando um vídeo é utilizado, como por exemplo, contar a quantidade de exibições de uma propanda na televisão. 
    
    Outra possibilidade de aplicação dessa técnica, de acordo com \citeauthor{chen2008video}, é encontrar pequenos trechos recortados de um vídeo e utilizados em outro, normalmente sem autorização do produtor original do conteúdo.

    Ainda segundo os autores, para que uma assinatura seja efetiva é necessário um descritor robusto, ou seja, capaz de gerar uma assinatura suficientemente semelhante para uma cópia que sofra distorções de enquadramento, variações de brilho, saturação, redimensionamento. O algoritmo também deve ser capaz de gerar assinaturas diferentes para vídeos com conteúdos diferentes. 
    
    teste \citeauthor{mao2015sceneframe}

    // TODO: ARRUMAR ESSE PARAGRAFO DEPOIS DE ARRUMAR A DEFINIÇÃO Pode-se separar os algoritmos de extração e descrição de vídeos em duas classes, sendo elas local e global. Descritores globais se baseiam em informações como histogramas, bordas e texturas presentes no vídeo e para \citeauthor{santos2004segmentaccao}, são considerados mais rudimentares.  Descritores locais são baseados em regiões e pontos de interesse como a utilização do SIFT (\textit{Scale-Invariant Feature Transform}) ou SURF (\textit{Speeded-Up Robust Features}). Nessa abordagem local, uma característica obtida é a repetibilidade, ou seja, mesmo que elementos no vídeo sejam alterados, transformados ou distorcidos, eles ainda representam as mesmas informações.
    
    Neste trabalho, foram implementados e comparados os seguintes algoritmos geradores de assinatura digital: 
    \begin{enumerate}
    \item Assinatura de vídeo baseada na medida ordinal.
    \end{enumerate}
    
    dando continuidade à monografia de \citeauthor{sylvio2015}, que estuda descritores globais. Por fim, foi projetada e implementada uma solução utilizando abordagens temporais e espaciais simultaneamente.
    

Nas Seções \ref{chap:definicoes}, \ref{chap:estadodaarte} e \ref{chap:relacionados} são apresentados conceitos importantes para o trabalho e é feita uma revisão do estado da arte com base em projetos da mesma área.


\section{Objetivos Gerais}
Este trabalho tem como objetivo comparar descritores para assinatura digital de vídeos.

\section{Objetivos Específicos}
\begin{itemize}
\item Implementar 6 algoritmos, locais e globais, para assinatura digital de vídeos.
\item Propor um novo descritor que combine características espaciais e temporais do vídeo.
\item Compilar uma base de vídeos para a realização dos testes e avaliação dos algoritmos.
\item Comparar os algoritmos com base na semelhança da assinatura e contrastar os métodos de comparação utilizados.
\end{itemize}
